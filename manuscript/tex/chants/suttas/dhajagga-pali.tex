\section{Dhaj'agga Sutta}

[Evam-me sutaṃ.] Ekaṃ samayaṃ Bhagavā, Sāvatthiyaṃ viharati, Jeta-vane
Anāthapiṇḍikassa ārāme. Tatra kho Bhagavā bhikkhū āmantesi: “bhikkhavo-ti”.
“Bhadante-ti,” te bhikkhū Bhagavato paccassosuṃ. Bhagavā etad avoca:

“Bhūta-pubbaṃ bhikkhave devāsura-saṅgāmo samupabbūḷho ahosi. Atha kho bhikkhave
Sakko devānamindo deve tāva-tiṃse āmantesi: ‘Sace mārisā devānaṃ saṅgāma-gatānaṃ
uppajjeyya bhayaṃ vā chambhitattaṃ vā lomahaṃso vā, mameva tasmiṃ samaye
dhaj’aggaṃ ullokeyyātha. Mamaṃ hi vo dhaj’aggaṃ ullokayataṃ yaṃ bhavissati
bhayaṃ vā chambhitattaṃ vā loma-haṃso vā, so pahīyissati.’

‘No ce me dhaj’aggaṃ ullokeyyātha, atha Pajāpatissa deva-rājassa dhaj’aggaṃ
ullokeyyātha. Pajāpatissa hi vo deva-rājassa dhaj’aggaṃ ullokayataṃ yaṃ
bhavissati bhayaṃ vā chambhitattaṃ vā loma-haṃso vā, so pahīyissati’.

‘No ce Pajāpatissa deva-rājassa dhaj’aggaṃ ullokeyyātha, atha Varuṇassa
deva-rājassa dhaj’aggaṃ ullokeyyātha. Varuṇassa hi vo deva-rājassa dha’jaggaṃ
ullokayataṃ yaṃ bhavissati bhayaṃ vā chambhitattaṃ vā lomahaṃso vā, so
pahīyissati’.

‘No ce Varuṇassa deva-rājassa dhaj’aggaṃ ullokeyyātha, atha Īsānassa
deva-rājassa dhaj’aggaṃ ullokeyyātha. Īsānassa hi vo devarājassa dhaj’aggaṃ
ullokayataṃ yaṃ bhavissati bhayaṃ vā chambhitattaṃ vā loma-haṃso vā, so
pahīyissatī-ti.’

“Taṃ kho pana bhikkhave Sakkassa vā devānam indassa dhaj’aggaṃ ullokayataṃ,
Pajāpatissa vā deva-rājassa dhaj’aggaṃ ullokayataṃ, Varuṇassa vā deva-rājassa
dhaj’aggaṃ ullokayataṃ, Īsānassa vā devarājassa dhaj’aggaṃ ullokayataṃ yaṃ
bhavissati bhayaṃ vā chambhitattaṃ vā loma-haṃso vā, so pahīyethāpi no’pi
pahīyetha.

“Taṃ kissa hetu? Sakko hi, bhikkhave, devānam indo avītarāgo avītadoso avītamoho
bhīru chambhī utrāsī palāyī-ti.

“Ahañ-ca kho, bhikkhave, evaṃ vadāmi: Sace tumhākaṃ, bhikkhave, arañña-gatānaṃ
vā rukkha-mūla-gatānaṃ vā suññāgāra-gatānaṃ vā uppajjeyya bhayaṃ vā
chambhitattaṃ vā loma-haṃso vā, mam eva tasmiṃ samaye anussareyyātha:

‘Iti pi so bhagavā arahaṃ sammā-sambuddho, vijjā-caraṇa-sampanno sugato
loka-vidū, anuttaro purisa-damma-sārathi satthā devamanussānaṃ Buddho
Bhagavā-ti. Mamaṃ hi vo bhikkhave anussarataṃ, yaṃ bhavissati bhayaṃ vā
chambhitattaṃ vā loma-haṃso vā, so pahīyissati.

“No ce maṃ anussareyyātha, atha dhammaṃ anussareyyātha:

‘Svākkhāto Bhagavatā dhammo, sandiṭṭhiko akāliko ehi-passiko, opanayiko
paccattaṃ veditabbo viññūhī-ti. Dhammaṃ hi vo bhikkhave anussarataṃ, yaṃ
bhavissati bhayaṃ vā chambhitattaṃ vā loma-haṃso vā, so pahīyissati.

“No ce dhammaṃ anussareyyātha, atha saṅghaṃ anussareyyātha:

‘Supaṭipanno Bhagavato sāvaka-saṅgho, uju-paṭipanno Bhagavato sāvaka-saṅgho,
ñāya-paṭipanno Bhagavato sāvaka-saṅgho, sāmīci-paṭipanno Bhagavato
sāvaka-saṅgho, yad-idaṃ cattāri purisa-yugāni aṭṭha purisapuggalā, esa Bhagavato
sāvaka-saṅgho, āhuneyyo pāhuneyyo dakkhiṇeyyo añjalikaraṇīyo, anuttaraṃ
puññakkhettaṃ lokassā-ti. Saṅghaṃ hi vo bhikkhave anussarataṃ yaṃ bhavissati
bhayaṃ vā chambhitattaṃ vā lomahaṃso vā, so pahīyissati.

“Taṃ kissa hetu? Tathāgato hi bhikkhave arahaṃ sammā-sambuddho, vītarāgo
vītadoso vītamoho, abhīru acchambhī anutrāsī apalāyīti.”

Idam avoca Bhagavā. Idaṃ vatvā sugato athāparaṃ etad avoca satthā:

“Araññe rukkha-mūle vā,\\
Suññ’āgāre va bhikkhavo;\\
Anussaretha Sambuddhaṃ,\\
Bhayaṃ tumhāka no siyā.\\
No ce Buddhaṃ sareyyātha,\\
Loka-jeṭṭhaṃ narāsabhaṃ;\\
Atha dhammaṃ sareyyātha,\\
Niyyānikaṃ sudesitaṃ.\\
No ce dhammaṃ sareyyātha,\\
Niyyānikaṃ sudesitaṃ;\\
Atha saṅghaṃ sareyyātha,\\
Puññakkhettaṃ anuttaraṃ.\\
Evaṃ-Buddhaṃ sarantānaṃ,\\
Dhammaṃ saṅghañ-ca bhikkhavo;\\
Bhayaṃ vā chambhitattaṃ vā,\\
Loma-haṃso na hessatī-ti.”

Dhaj’agga Suttaṃ Niṭṭhitaṃ.

\suttaRef{S.I.218}

