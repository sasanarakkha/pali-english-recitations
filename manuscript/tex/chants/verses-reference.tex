\chapter{Verses}

\section{The Buddha's First Exclamation}
\paliTitle{Buddha-paṭhama-bhāsita}

\begin{twochants}
  Aneka-jāti-saṁsāraṁ & Sandhāvissaṁ anibbisaṁ\\
  Gaha-kāraṁ gavesanto & Dukkhā jāti punappunaṁ\\
\end{twochants}

\begin{english}
  For many lifetimes in the round of birth\\
  Wandering on endlessly\\
  For the builder of this house I searched\\
  How painful is repeated birth.
\end{english}

\begin{twochants}
  Gaha-kāraka diṭṭho'si & Puna gehaṁ na kāhasi\\
  Sabbā te phāsukā bhaggā & Gaha-kūṭaṁ visaṅkhataṁ\\
  Visaṅkhāra-gataṁ cittaṁ & Taṇhānaṁ khayam-ajjhagā\\
\end{twochants}

\begin{english}
  House-builder you've been seen\\
  Another home you will not build\\
  All your rafters have been snapped\\
  Dismantled is your ridge-pole\\
  The non-constructing mind\\
  Has come to craving's end
\end{english}

\suttaRef{Dhp 153-154}

\clearpage

\section{Respect for the Dhamma}
\paliTitle{Dhamma-gārava}

\begin{twochants}
  Ye ca atītā sambuddhā & Ye ca buddhā anāgatā \\
  Yo c'etarahi sambuddho & Bahunnaṃ soka-nāsano \\
\end{twochants}

\begin{english}
  All the Buddhas of the past\\
  All the Buddhas yet to come\\
  The Buddha of this current age\\
  Dispellers of much sorrow
\end{english}

\begin{twochants}
  Sabbe saddhamma-garuno & Vihariṃsu viharanti ca\\
  Atho pi viharissanti & Esā buddhāna dhammatā\\
\end{twochants}

\begin{english}
  Those having lived or living now\\
  Those living in the future\\
  All do revere the True Dhamma\\
  That is the nature of all Buddhas
\end{english}

\begin{twochants}
  Tasmā hi atta-kāmena & Mahattam-abhikaṅkhatā\\
  Saddhammo garu-kātabbo & Saraṃ buddhāna sāsanaṃ\\
\end{twochants}

\begin{english}
  Therefore desiring one's own welfare\\
  Pursuing greatest aspirations\\
  One should revere the True Dhamma\\
  Recollecting the Buddha's teaching
\end{english}

\suttaRef{SN 6.2}

\begin{paritta}
  Na hi dhammo adhammo ca\\
  Ubho sama-vipākino\\
  Adhammo nirayaṃ neti\\
  Dhammo pāpeti suggatiṃ
\end{paritta}

\begin{english}
  What is true Dhamma and what's not\\
  Will never have the same results\\
  While wrong Dhamma leads to hell realms\\
  True Dhamma takes one on a good course
\end{english}

\begin{paritta}
  Dhammo have rakkhati dhamma-cāriṃ\\
  Dhammo suciṇṇo sukham-āvahāti\\
  Esānisaṃso dhamme suciṇṇe\\
  Na duggatiṃ gacchati dhamma-cārī
\end{paritta}

\clearpage

\begin{english}
  The Dhamma guards those who live in line with it\\
  And leads to happiness when practised well\\
  This is the blessing of well-practised Dhamma\\
  The Dhamma-farer does not go on a bad course
\end{english}

\suttaRef{Thag 4.10}

\clearpage

\section{Going to True and False Refuges}
\paliTitle{Khemākhema-saraṇa-gamana}

\begin{twochants}
Bahuṃ ve saraṇaṃ yanti & Pabbatāni vanāni ca\\
Ārāma-rukkha-cetyāni & Manussā bhaya-tajjitā\\
\end{twochants}

\begin{english}
  To many refuges they go\\
  To mountain slopes and forest glades\\
  To parkland shrines and sacred sites\\
  People overcome by fear
\end{english}

\begin{twochants}
N'etaṃ kho saraṇaṃ khemaṃ & N'etaṃ saraṇam-uttamaṃ\\
N'etaṃ saraṇam-āgamma & Sabba-dukkhā pamuccati\\
\end{twochants}

\begin{english}
  Such a refuge is not secure\\
  Such a refuge is not supreme\\
  Such a refuge does not bring\\
  Complete release from all suffering
\end{english}

\begin{twochants}
Yo ca buddhañ-ca dhammañ-ca & Saṅghañ-ca saraṇaṃ gato\\
Cattāri ariya-saccāni & Sammappaññāya passati\\
\end{twochants}

\begin{english}
  Whoever goes to refuge\\
  In the Triple Gem\\
  Sees with right discernment\\
  The Four Noble Truths
\end{english}

\begin{twochants}
Dukkhaṃ dukkha-samuppādaṃ & Dukkhassa ca atikkamaṃ\\
Ariyañ-c'aṭṭh'aṅgikaṃ maggaṃ & Dukkhūpasama-gāminaṃ\\
\end{twochants}

\begin{english}
  Suffering and its origin\\
  And that which lies beyond\\
  The Noble Eightfold Path\\
  That leads the way to suffering's end.
\end{english}

\begin{twochants}
Etaṃ kho saraṇaṃ khemaṃ & Etaṃ saraṇam-uttamaṃ\\
Etaṃ saraṇam-āgamma & Sabba-dukkhā pamuccatī'ti.
\end{twochants}

\begin{english}
  Such a refuge is secure\\
  Such a refuge is supreme\\
  Such a refuge truly brings\\
  Complete release from all suffering.
\end{english}

\suttaRef{Dhp 188-192}

\section{The Pāṭimokkha Exhortation}
\paliTitle{Ovāda-pāṭimokkha-gāthā}

\begin{leader}
  [Handa mayaṃ ovāda-pāṭimokkha-gāthāyo bhaṇāmase]
\end{leader}

Sabba-pāpassa akaraṇaṃ

\begin{cprenglish}
  Not doing any evil
\end{cprenglish}

Kusalassūpasampadā

\begin{cprenglish}
  To be committed to the good
\end{cprenglish}

Sacitta-pariyodapanaṃ

\begin{cprenglish}
  To purify one's mind
\end{cprenglish}

Etaṃ buddhāna sāsanaṃ

\begin{cprenglish}
  These are the teachings of all Buddhas
\end{cprenglish}

Khantī paramaṃ tapo tītikkhā

\begin{cprenglish}
  Patient endurance is the highest practice burning out defilements
\end{cprenglish}

Nibbānaṃ paramaṃ vadanti buddhā

\begin{cprenglish}
  The Buddhas say Nibbāna is supreme
\end{cprenglish}

Na hi pabbajito parūpaghātī

\begin{cprenglish}
  Not a renunciant is one who injures others
\end{cprenglish}

Samaṇo hoti paraṃ viheṭhayanto

\begin{cprenglish}
  Whoever troubles others can't be called a monk
\end{cprenglish}

Anūpavādo anūpaghāto

\begin{cprenglish}
  Not to insult and not to injure
\end{cprenglish}

Pāṭimokkhe ca saṃvaro

\begin{cprenglish}
  To live restrained by training rules
\end{cprenglish}

Mattaññutā ca bhattasmiṃ

\begin{cprenglish}
  Knowing one's measure at the meal
\end{cprenglish}

Pantañca sayan'āsanaṃ

\begin{cprenglish}
  Retreating to a lonely place
\end{cprenglish}

Adhicitte ca āyogo

\begin{cprenglish}
  Devotion to the higher mind
\end{cprenglish}

Etaṃ buddhāna sāsanaṃ

\begin{cprenglish}
  These are the teachings of all Buddhas
\end{cprenglish}

  \suttaRef{Dhp 183-185}
