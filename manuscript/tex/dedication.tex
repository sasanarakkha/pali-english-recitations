\cleartorecto
\thispagestyle{empty}

{\setlength{\parskip}{10pt}

{\centering\fontsize{16}{25}\selectfont
\textsc{We Wish To Gratefully Acknowledge}
\par}

The Saṅghas of Wat Pah Nanachat (WPN), Amaravati, and Abhayagiri for allowing the use of material from their respective chanting books, the late Ven. Dr. Saddhātissa and Mr. Maurice Walshe for their English translations, as well as Ven. Bhikkhu Bodhi for granting permission to use and slightly adapt his translations. Various Saṅgha members of SBS Monk Training Centre, who contributed in the compilation of an interesting selection of chants, as well as for providing countless suggestions to help improve the English translations.

\linkdest{endnote1-body}
Additional information on translations, as well as deviations\makeatletter\hyperlink{endnote1-appendix}\Hy@raisedlink{{\pagenote{%
  \hypertarget{endnote1-appendix}{\hyperlink{endnote1-body}{Due to the balanced and inspiring selection of chants, as well as for the sake of compatability, WPN \textit{Buddhist Chanting} {2014} has served as the basis for this book. Over time, suggestions for the inclusion of additional chants, as well as occasional improvements of existing translations were incorporated. Such changes were meticulously marked down in the endnotes, so that someone familiar with the present book can straight away find the relevant differences, which can be useful when visiting a branch monastery of the Ajahn Chah lineage, in order to know in which places to revert to the original version.}}}}}\makeatother\thickspace
from WPN \textit{Buddhist Chanting} (2014), have been annotated by Ven. Ariyadhammika in the endnotes.

\medskip

{\centering
To Āyasmā Aggacitta, the founding father of\\
Sāsanārakkha Buddhist Sanctuary.

\medskip

\includegraphics[height=54mm]{sbs-logo-tuck-loon-brown.jpg}

}

}
