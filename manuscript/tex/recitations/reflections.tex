\chapterOpeningPage{reflections.pdf}

\chapter{Reflections}

\sectionPaliTitle{Cattaro parrikhārā}
\section{The Four Requisites}
\label{four-requisites}

\begin{leader}
  〈 Handa mayaṁ taṅkhaṇika-paccavekkhaṇa-pāṭhaṁ 〉

\end{leader}

Paṭisaṅkhā yoniso cīvaraṁ paṭisevāmi\\
Yāvadeva sītassa paṭighātāya\\
Uṇhassa paṭighātāya\\
Ḍaṁsa-makasa-vātātapa-siriṁsapa-samphassānaṁ paṭighātāya\\
Yāvadeva hirikopina-paṭicchādanatthaṁ

\begin{english-verses}
  Wisely reflecting \breathmark\ I use the robe\\
  % TODO: hang does not work well here, leaves too much space b4 and after
  Only to ward off cold \breathmark\ to ward off heat \breathmark\ to ward off the touch of \mbox{flies}~\breathmark\ mosquitoes wind burning and creeping things\\
  Only for the sake of modesty
\end{english-verses}

Paṭisaṅkhā yoniso piṇḍapātaṁ paṭisevāmi\\
Neva davāya na madāya na maṇḍanāya na vibhūsanāya\\
Yāvadeva imassa kāyassa ṭhitiyā yāpanāya\\
Vihiṁsūparatiyā brahmacariyānuggahāya\\
Iti purāṇañca vedanaṁ paṭihaṅkhāmi\\
Navañca vedanaṁ na uppādessāmi\\
Yātrā ca me bhavissati anavajjatā ca phāsuvihāro cā'ti

\begin{english-verses}
  Wisely reflecting \breathmark\ I use almsfood\\
  Not for fun \breathmark\ not for pleasure \breathmark\ not for fattening \breathmark\ not for beautification\\
  Only for the maintenance and nourishment of this body\\
  For keeping it healthy \breathmark\ for helping with the holy life\\
  Thinking thus: ``I will allay hunger without overeating\\
  So that I may continue to live blamelessly and at ease''
\end{english-verses}

Paṭisaṅkhā yoniso senāsanaṁ paṭisevāmi\\
Yāvadeva sītassa paṭighātāya\\
Uṇhassa paṭighātāya\\
Ḍaṁsa-makasa-vātātapa-siriṁsapa-samphassānaṁ paṭighātāya\\
Yāvadeva utuparissaya-vinodanaṁ paṭisallānārāmatthaṁ

\begin{english-verses}
  Wisely reflecting \breathmark\ I use the lodging\\
  Only to ward off cold \breathmark\ to ward off heat \breathmark\ to ward off the touch of flies \breathmark\ mosquitoes wind burning and creeping things\\
  Only to remove the danger from weather \breathmark\ and for living in seclusion
\end{english-verses}

Paṭisaṅkhā yoniso gilāna-paccaya-bhesajja-parikkhāraṁ paṭisevāmi\\
Yāvadeva uppannānaṁ veyyābādhikānaṁ vedanānaṁ paṭighātāya\\
Abyāpajjha-paramatāyā'ti

\begin{english-verses}
  Wisely reflecting \breathmark\ I use supports for the sick and medicinal requisites\\
  Only to ward off painful feelings that have arisen\\
  For the maximum freedom from disease
\end{english-verses}

\suttaRef{[MN 2]}

\bottomNav{five-recollections}

\sectionPaliTitle{Āhāra-paṭikūla-paccavekkhaṇa-pāṭho}
\section{The Repulsiveness of Food}
\label{repulsiveness-of-food}

\begin{leader}
  〈 Handa mayaṁ āhāra-paṭikūla-paccavekkhaṇa-pāṭhaṁ 〉

\end{leader}

\begin{pali-hang}
Āhāre paṭikūlasaññāparicitena bhikkhave \breathmark\ bhikkhuno cetasā bahulaṁ viharato
\end{pali-hang}

\begin{english}
  When a bhikkhu often dwells with a mind\\
  Accustomed to the perception of the repulsiveness of food
\end{english}

Rasataṇhāya cittaṁ patilīyati

\begin{english}
  His mind shrinks away from craving for tastes
\end{english}

Patikuṭati pativattati na sampasāriyati

\begin{english}
  Turns back from it\\
  Rolls away from it\\
  And is not drawn towards it
\end{english}

Upekkhā vā pāṭikulyatā vā saṇṭhāti

\begin{english}
  Either equanimity or disgust become settled in him
\end{english}

\suttaRef{[AN 7.49]}

Sabbo panāyaṁ piṇḍa-pāto ajigucchanīyo

\begin{english}
  None of this almsfood is innately repulsive
\end{english}

Imaṁ pūti-kāyaṁ patvā

\begin{english}
  But touching this unclean body
\end{english}

Ativiya jigucchanīyo jāyati

\begin{english}
  It becomes disgusting indeed
\end{english}

\suttaRef{[Trad]}

\bottomNav{requisites-for-awakening}

\sectionPaliTitle{Mettā-pharaṇa}
\section{Universal Well-Being}
\label{universal-well-being}

\begin{leader}
  〈 Handa mayaṁ mettāpharaṇaṁ karomase 〉

\end{leader}

Ahaṁ sukhito homi\\
Niddukkho homi\\
Avero homi\\
Abyāpajjho homi\\
Anīgho homi\\
Sukhī attānaṁ pariharāmi\\
Sabbe sattā sukhitā hontu\\
Sabbe sattā averā hontu\\
Sabbe sattā abyāpajjhā hontu\\
Sabbe sattā anīghā hontu\\
Sabbe sattā sukhī attānaṁ pariharantu\\
Sabbe sattā sabbadukkhā pamuccantu\\
Sabbe sattā laddha-sampattito mā vigacchantu

Sabbe sattā kammassakā kammadāyādā kammayonī kammabandhū kammapaṭisaraṇā\\
Yaṁ kammaṁ karissanti\\
Kalyāṇaṁ vā pāpakaṁ vā\\
Tassa dāyādā bhavissanti

\clearpage

\begin{leader}
  〈 Now let us recite the reflections on universal well-being 〉
\end{leader}

\begin{english-verses}
  May I abide in well-being\\
  In freedom from affliction\\
  In freedom from hostility\\
  In freedom from ill-will\\
  In freedom from anxiety\\
  And may I maintain well-being in myself\\
  May everyone abide in well-being\\
  In freedom from hostility\\
  In freedom from ill-will\\
  In freedom from anxiety\\
  And may they maintain well-being in themselves\\
  May all beings be released from all suffering\\
\end{english-verses}
\begin{english-hang}
  And may they not be parted from the good fortune they have attained\pagenote{%
    In the original version, this line is followed by ``When they act upon intention'', which is not found in the Pāli, and is potentially misleading, giving the implication that intention alone is not enough to count as kamma.}
\end{english-hang}
% TODO: hang doesn't look good here
\begin{english-verses}
  All beings are the owners of their kamma\pagenote{%
    Orig: ``All beings are the owners of their action and inherit its results. Their future is born from such action, companion to such action, and its results will be their home. All actions with intention, be they skilful or harmful, of such acts they will be the heirs.'' For the sake of consistency with other chants within this chanting book, the original version was substituted with the one found in the ``Five subjects for frequent recollection'', and ``Ten subjects for frequent recollection by one who has gone forth''.}\\
  Heirs to their kamma\\
  Born of their kamma\\
  Related to their kamma\\
  Abide supported by their kamma\\
  Whatever kamma they shall do\\
  Either skillful or harmful\\
  Of such acts \breathmark\ they will be the heirs\\
\end{english-verses}

\suttaRef{[AN 3.65 \& 5.57]}

\bottomNav{seven-factors-of-awakening}

\sectionPaliTitle{Brahmavihārā}
\section{The Divine Abidings}
\label{divine-abidings}

\begin{leader}
  〈 Handa mayaṁ caturappamaññā obhāsanaṁ karomase 〉

\end{leader}

\begin{pali-hang}
Mettā-sahagatena cetasā ekaṁ disaṁ pharitvā viharati tathā dutiyaṁ tathā tatiyaṁ tathā catutthaṁ iti uddhamadho tiriyaṁ sabbadhi sabbattatāya sabbāvantaṁ lokaṁ mettā-sahagatena cetasā vipulena mahaggatena appamāṇena averena abyāpajjhena pharitvā viharati
\end{pali-hang}

\medskip

\begin{pali-hang}
Karuṇā-sahagatena cetasā ekaṁ disaṁ pharitvā viharati tathā dutiyaṁ tathā tatiyaṁ tathā catutthaṁ iti uddhamadho tiriyaṁ sabbadhi sabbattatāya sabbāvantaṁ lokaṁ karuṇā-sahagatena cetasā vipulena mahaggatena appamāṇena averena abyāpajjhena pharitvā viharati
\end{pali-hang}

\medskip

\begin{pali-hang}
Muditā-sahagatena cetasā ekaṁ disaṁ pharitvā viharati tathā dutiyaṁ tathā tatiyaṁ tathā catutthaṁ iti uddhamadho tiriyaṁ sabbadhi sabbattatāya sabbāvantaṁ lokaṁ muditā-sahagatena cetasā vipulena mahaggatena appamāṇena averena abyāpajjhena pharitvā viharati
\end{pali-hang}

\medskip

\begin{pali-hang}
Upekkhā-sahagatena cetasā ekaṁ disaṁ pharitvā viharati tathā dutiyaṁ tathā tatiyaṁ tathā catutthaṁ iti uddhamadho tiriyaṁ sabbadhi sabbattatāya sabbāvantaṁ lokaṁ upekkhā-sahagatena cetasā vipulena mahaggatena appamāṇena averena abyāpajjhena pharitvā viharatī'ti
\end{pali-hang}

\clearpage

\begin{leader}
  〈 Now let us make the Four Boundless Qualities shine forth 〉
\end{leader}

\begin{english-hang}
I will abide pervading one quarter with a heart imbued with loving-kindness\\
\end{english-hang}

\begin{english}
Likewise the second likewise the third likewise the fourth\\
So above and below around and everywhere and to all as to myself\\
\end{english}

\begin{english-hang}
I will abide pervading the all-encompassing world with a heart imbued with loving-kindness\\
\end{english-hang}

\begin{english-hang}
Abundant exalted immeasurable without hostility and without ill-will
\end{english-hang}

\medskip

\begin{english-hang}
I will abide pervading one quarter with a heart imbued with compassion\\
\end{english-hang}

\begin{english}
Likewise the second likewise the third likewise the fourth\\
So above and below around and everywhere and to all as to myself\\
\end{english}

\begin{english-hang}
I will abide pervading the all-encompassing world with a heart imbued with compassion\\
\end{english-hang}

\begin{english-hang}
Abundant exalted immeasurable without hostility and without ill-will
\end{english-hang}

\medskip

\begin{english-hang}
I will abide pervading one quarter with a heart imbued with gladness\pagenote{%
  Orig: ``a heart imbued with gladness''}\\
\end{english-hang}

\begin{english}
Likewise the second likewise the third likewise the fourth\\
So above and below around and everywhere and to all as to myself\\
\end{english}

\begin{english-hang}
I will abide pervading the all-encompassing world with a heart imbued with gladness\\
\end{english-hang}

\begin{english-hang}
Abundant exalted immeasurable without hostility and without ill-will
\end{english-hang}

\medskip

\begin{english-hang}
I will abide pervading one quarter with a heart imbued with equanimity\\
\end{english-hang}

\begin{english}
Likewise the second likewise the third likewise the fourth\\
So above and below around and everywhere and to all as to myself\\
\end{english}

\begin{english-hang}
I will abide pervading the all-encompassing world with a heart imbued with equanimity\\
\end{english-hang}

\begin{english-hang}
Abundant exalted immeasurable without hostility and without ill-will
\end{english-hang}

\suttaRef{[DN 13]}

\bottomNav{ten-recollections}

\sectionPaliTitle{Pañca-abhiṇha-paccavekkhaṇā}
\section{Five Subjects for Frequent Recollection}
\label{five-recollections}

\begin{leader}
  〈 Handa mayaṁ abhiṇha-paccavekkhaṇa-pāṭhaṁ 〉

\end{leader}

Jarā-dhammomhi jaraṁ anatīto

\begin{english}
  I am of the nature to age\\
  I have not gone beyond ageing
\end{english}

Byādhi-dhammomhi byādhiṁ anatīto

\begin{english}
  I am of the nature to sicken\\
  I have not gone beyond sickness
\end{english}

Maraṇa-dhammomhi maraṇaṁ anatīto

\begin{english}
  I am of the nature to die\\
  I have not gone beyond dying
\end{english}

Sabbehi me piyehi manāpehi nānābhāvo vinābhāvo

\begin{english}
  All that is mine beloved and pleasing\\
  Will become otherwise\\
  Will become separated from me
\end{english}

\begin{pali-hang}
Kammassakomhi kammadāyādo kammayoni kammabandhu kammapaṭisaraṇo\\
\end{pali-hang}
% TODO: hang indent looks weird here
Yaṁ kammaṁ karissāmi\\
Kalyāṇaṁ vā pāpakaṁ vā\\
Tassa dāyādo bhavissāmi

\begin{english}
  I am the owner of my kamma\\
  Heir to my kamma\\
  Born of my kamma\\
  Related to my kamma\\
  Abide supported by my kamma\\
  Whatever kamma I shall do\\
  Either skillful or harmful\\
  Of such acts \breathmark\ I will be the heir\pagenote{%
    Orig: `For good or for ill, Of that I will be the heir' For the sake of consistency with other passages which were translated differently, that translation has been chosen.}
\end{english}

Evaṁ amhehi abhiṇhaṁ paccavekkhitabbaṁ

\begin{english}
  Thus we should frequently recollect
\end{english}

\suttaRef{[AN 5.57]}

\bottomNav{32-parts}

\sectionPaliTitle{Dasadhammā pabbajita-abhiṇha-paccavekkhaṇā}
\section{Ten Subjects for Frequent Reflection by One Gone Forth}
\label{ten-recollections}

\begin{leader}
  〈 Handa mayaṁ pabbajita-abhiṇha-paccavekkhaṇa-pāṭhaṁ 〉

\end{leader}

Dasa ime bhikkhave dhammā\\
Pabbajitena abhiṇhaṁ paccavekkhitabbā\\
Katame dasa

\begin{english-hang}
  Bhikkhus there are these ten dhammas\pagenote{%
    Orig: ``Bhikkhus there are ten dhammas''}
  \breathmark\ which should be reflected upon again and again by one who has gone forth\\
\end{english-hang}

\begin{english}
  What are these ten?
\end{english}

\begin{pali-hang}
Vevaṇṇiyamhi ajjhūpagato'ti pabbajitena abhiṇhaṁ paccavekkhitabbaṁ
\end{pali-hang}

\begin{english}
  ``I have reached a state of castelessness''\pagenote{%
    Orig: ``I am no longer living according to worldly aims and values''}\\
\end{english}

\begin{english-hang}
  This should be reflected upon again and again by one who has gone forth
\end{english-hang}

Parapaṭibaddhā me jīvikā'ti\\
Pabbajitena abhiṇhaṁ paccavekkhitabbaṁ

\begin{english}
  ``My very life is sustained through the gifts of others''\\
\end{english}

\begin{english-hang}
  This should be reflected upon again and again by one who has gone forth
\end{english-hang}

Añño me ākappo karaṇīyo'ti\\
Pabbajitena abhiṇhaṁ paccavekkhitabbaṁ

\begin{english}
  ``Now my conduct should be different from before''\pagenote{%
    Orig: `I should strive to abandon my former habits'}\\
\end{english}

\begin{english-hang}
  This should be reflected upon again and again by one who has gone forth
\end{english-hang}

Kacci nu kho me attā sīlato na upavadatī'ti\\
Pabbajitena abhiṇhaṁ paccavekkhitabbaṁ

\begin{english}
  ``Does regret over my conduct arise in my mind?''\\
\end{english}

\begin{english-hang}
  This should be reflected upon again and again by one who has gone forth
\end{english-hang}

\begin{pali-hang}
Kacci nu kho maṁ anuvicca viññū sabrahmacārī sīlato na upavadantī'ti\\
\end{pali-hang}
Pabbajitena abhiṇhaṁ paccavekkhitabbaṁ

\begin{english}
  ``Could my spiritual companions find fault with my conduct?''\\
\end{english}

\begin{english-hang}
  This should be reflected upon again and again by one who has gone forth
\end{english-hang}

Sabbehi me piyehi manāpehi nānābhāvo vinābhāvo'ti\\
Pabbajitena abhiṇhaṁ paccavekkhitabbaṁ

\begin{english}
  ``All that is mine beloved and pleasing\\
  Will become otherwise\\
  Will become separated from me''\\
\end{english}

\begin{english-hang}
  This should be reflected upon again and again by one who has gone forth
\end{english-hang}

\begin{pali-hang}
Kammassakomhi kammadāyādo kammayoni kammabandhu kammapaṭisaraṇo\\
\end{pali-hang}
% TODO: hang is off here
Yaṁ kammaṁ karissāmi\\
Kalyāṇaṁ vā pāpakaṁ vā\\
Tassa dāyādo bhavissāmī'ti\\
Pabbajitena abhiṇhaṁ paccavekkhitabbaṁ

\begin{english}
  ``I am the owner of my kamma\\
  Heir to my kamma\\
  Born of my kamma\\
  Related to my kamma\\
  Abide supported by my kamma\\
  Whatever kamma I shall do\\
  Either skillful or harmful\\
  Of such acts \breathmark\ I will be the heir''\pagenote{%
    Orig: ``For good or for ill, Of that I will be the heir`` For the sake of consistency with other passages which were translated differently, this translation has been chosen.}\\
\end{english}

\begin{english-hang}
  This should be reflected upon again and again by one who has gone forth
\end{english-hang}

`Kathambhūtassa me rattindivā vītipatantī'ti\\
Pabbajitena abhiṇhaṁ paccavekkhitabbaṁ

\begin{english}
  ``The days and nights are relentlessly passing\\
  How well am I spending my time?''\\
\end{english}

\begin{english-hang}
  This should be reflected upon again and again by one who has gone forth
\end{english-hang}

Kacci nu kho'haṁ suññāgāre abhiramāmī'ti\\
Pabbajitena abhiṇhaṁ paccavekkhitabbaṁ

\begin{english}
  ``Do I delight in solitude or not?''\\
\end{english}

\begin{english-hang}
  This should be reflected upon again and again by one who has gone forth by one who has gone forth
\end{english-hang}

\begin{pali-hang}
Atthi nu kho me uttari-manussa-dhammā alamariya-ñāṇa-dassana viseso adhigato\\
\end{pali-hang}

\begin{pali-hang}
So'haṁ pacchime kāle sabrahmacārīhi puṭṭho na maṅku bhavissāmī'ti\\
\end{pali-hang}
% TODO: hang is off here
\begin{pali-hang}
Pabbajitena abhiṇhaṁ paccavekkhitabbaṁ
\end{pali-hang}

\begin{english}
  ``Has my practice borne fruit with freedom or insight\\
  So that at the end of my life \breathmark\ I need not feel ashamed when questioned by my spiritual companions?''\\
\end{english}

\begin{english-hang}
  This should be reflected upon again and again by one who has gone forth
\end{english-hang}

Ime kho bhikkhave dasa dhammā\\
Pabbajitena abhiṇhaṁ paccavekkhitabbā'ti

\begin{english-hang}
  Bhikkhus these are the ten dhammas \breathmark\ which should be reflected upon again and again by one who has gone forth
\end{english-hang}

\suttaRef{[AN 10.48]}

\bottomNav{sharing-aspirations}

\sectionPaliTitle{Dvattiṁsākārapaccavekkhaṇa}
\section{The Thirty-Two Parts}
\label{32-parts}

\begin{leader}
  〈 Handa mayaṁ dvattiṁsākāra-pāṭhaṁ 〉

\end{leader}

\begin{pali-hang}
Ayaṁ kho me kāyo uddhaṁ pādatalā adho kesamatthakā tacapariyanto pūro nānappakārassa asucino
\end{pali-hang}

\begin{english-verses}
  This which is my body\\
  From the soles of the feet up\\
  And down from the crown of the head\\
  Is a sealed bag of skin\\
  Filled with unattractive things
\end{english-verses}

Atthi imasmiṁ kāye

\begin{english}
  In this body there are
\end{english}

{\centering
  \setArrayStretch{1}

  \begin{tabular}{ r l }
    kesā            & \tr{hair of the head} \\
    lomā            & \tr{hair of the body} \\
    nakhā           & \tr{nails} \\
    dantā           & \tr{teeth} \\
    taco            & \tr{skin} \\
  \end{tabular}

  \begin{tabular}{ r l }
    maṁsaṁ          & \tr{flesh} \\
    nahārū          & \tr{sinews} \\
    aṭṭhī           & \tr{bones} \\
    aṭṭhimiñjaṁ     & \tr{bone marrow} \\
    vakkaṁ          & \tr{kidneys} \\
    hadayaṁ         & \tr{heart} \\
    yakanaṁ         & \tr{liver} \\
    kilomakaṁ       & \tr{membranes} \\
    pihakaṁ         & \tr{spleen} \\
    papphāsaṁ       & \tr{lungs} \\
    antaṁ           & \tr{bowels} \\
    antaguṇaṁ       & \tr{entrails} \\
    udariyaṁ        & \tr{undigested food} \\
    karīsaṁ         & \tr{excrement} \\
    pittaṁ          & \tr{bile} \\
    semhaṁ          & \tr{phlegm} \\
    pubbo           & \tr{pus} \\
    lohitaṁ         & \tr{blood} \\
    sedo            & \tr{sweat} \\
    medo            & \tr{fat} \\
    assu            & \tr{tears} \\
    vasā            & \tr{grease} \\
    kheḷo           & \tr{spittle} \\
    siṅghāṇikā      & \tr{mucus} \\
    lasikā          & \tr{oil of the joints} \\
    muttaṁ          & \tr{urine} \\
    matthaluṅgan'ti & \tr{brain}\pagenote{%
                      In the discourses, except for one occasion in the Khp, the brain is not mentioned as a separate organ or body part, making it a list of only 31 body parts.} \\
  \end{tabular}

  \restoreArrayStretch
}

\begin{pali-hang}
Evam-ayaṁ me kāyo uddhaṁ pādatalā adho kesamatthakā tacapariyanto pūro nānappakārassa asucino
\end{pali-hang}

\begin{english-verses}
  This then which is my body\\
  From the soles of the feet up\\
  And down from the crown of the head\\
  Is a sealed bag of skin\\
  Filled with unattractive things
\end{english-verses}



\suttaRef{[DN 22]}

\bottomNav{principles-of-cordiality}

\sectionPaliTitle{Aniccānussati}
\section{Recollection of Impermanence}
\label{recollection-impermanence}

\begin{leader}
  〈 Handa mayaṁ aniccānussati-pāṭhaṁ 〉

\end{leader}

Sabbe saṅkhārā anicca

\begin{english}
  All conditioned things are impermanent
\end{english}

Sabbe saṅkhārā dukkhā

\begin{english}
  All conditioned things are dukkha
\end{english}

Sabbe dhammā anattā

\begin{english}
  All things are not-self\pagenote{%
    Orig: ``Everything is void of self''}
\end{english}

\suttaRef{[Dhp 277-279]}

Addhuvaṁ jīvitaṁ

\begin{english}
  Life is not for sure
\end{english}

Dhuvaṁ maraṇaṁ

\begin{english}
  Death is for sure
\end{english}

Avassaṁ mayā maritabbaṁ

\begin{english}
  It is inevitable that I'll die
\end{english}

Maraṇa-pariyosānaṁ me jīvitaṁ

\begin{english}
  Death is the culmination of my life
\end{english}

Jīvitaṁ me aniyataṁ

\begin{english}
  My life is uncertain
\end{english}

Maraṇaṁ me niyataṁ

\begin{english}
  My death is certain
\end{english}

\suttaRef{[Dhp A]}

Vata

\begin{english}
  Indeed
\end{english}

Ayaṁ kāyo

\begin{english}
  This body
\end{english}

Aciraṁ

\begin{english}
  Will soon
\end{english}

Apeta-viññāṇo

\begin{english}
  Be void of consciousness
\end{english}

Chuḍḍho

\begin{english}
  And cast away
\end{english}

Adhisessati

\begin{english}
  It will lie
\end{english}

Paṭhaviṁ

\begin{english}
  On the ground
\end{english}

Kaliṅgaraṁ iva

\begin{english}
  Just like a rotten log
\end{english}

Niratthaṁ

\begin{english}
  Useless\pagenote{%
    Orig: ``Completely void of use''}
\end{english}

\suttaRef{[Dhp 41]}

Aniccā vata saṅkhārā

\begin{english}
  Indeed \breathmark\ conditioned things cannot last
\end{english}

Uppāda-vaya-dhammino

\begin{english}
  Their nature is to rise and cease\pagenote{%
    Orig: ``Their nature is to rise and fall''}
\end{english}

Uppajjitvā nirujjhanti

\begin{english}
  Having arisen things must cease
\end{english}

Tesaṁ vūpasamo sukho

\begin{english}
  Their stilling is true happiness
\end{english}

\suttaRef{[Trad]}

\bottomNav{yatha-vari-vaha-pura}
